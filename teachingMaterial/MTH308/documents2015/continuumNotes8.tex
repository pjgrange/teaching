%\documentclass[12pt]{article}
%\documentclass[a4paper, 12pt]{scrartcl}
%\voffset=-0.9in
%\hoffset=-0.6in
\documentclass[DIV=12]{article}
\setlength{\textheight}{9.1in}
\setlength{\textwidth}{7in} 
\usepackage[margin=1.1in]{geometry}
%\usepackage[justification=justified,singlelinecheck=off]{caption}


%\cofoot{\footnotesize{\rm{Pascal Grange (August 2013)}} {\rm{\ttfamily{ pascal.grange@polytechnique.org}}}}


\let\oldbibliography\thebibliography
\renewcommand{\thebibliography}[1]{%
  \oldbibliography{#1}%
  \setlength{\itemsep}{-1pt}%
{\small
\bibliography{bibfile}}
}


\usepackage[dvips]{color}
\bibliographystyle{plain}

\usepackage{graphicx}
%\usepackage{pdfpages}
%\usepackage{multicol}
%\usepackage{pstricks,pst-grad}
%\usepackage{epsfig}
\usepackage{amsmath,esint,amssymb}

%\usepackage{subfig}
%\usepackage{tikz}
%\usepackage{verbatim}
\usepackage{amsmath}

%\usepackage{array}
%\pagestyle{scrheadings}

\newcommand*\Laplace{\mathop{}\!\mathbin\bigtriangleup}
\newcommand{\eBase}{(\vec{e_1}, \vec{e_2},\vec{e_3})}
\newcommand{\eBasePrime}{\left(\vec{e'_1}, \vec{e'_2},\vec{e'_3}\right)}
\newcommand{\maxX}{12}
\newcommand{\maxY}{12}
\newcommand{\vol}{\mathcal{V}}
\newcommand{\surf}{{\mathcal{S}}}
\newcommand{\sExt}{{{\mathcal{S}}\rightarrow{\mathrm{ext}}}}
\newcommand{\intVol}{\iiint}
\newcommand{\intSurf}{\oiint}
\newcommand{\fVol}{\vec{f^{vol}}}
\newcommand{\kg}{\mathrm{kg}}
\newcommand{\m}{\mathrm{m}}
\newcommand{\cm}{\mathrm{cm}}
\newcommand{\s}{\mathrm{s}}
\newcommand{\Pa}{\mathrm{Pa}}
\newcommand{\GPa}{\mathrm{GPa}}
\newcommand{\Tr}{\mathrm{Tr}}
\newcommand{\whatIf}{\operatornamewithlimits{=}}



\begin{document}

\title{
\noindent\hrulefill
\begin{flushleft}
{\Large \bf{XJTLU, MTH308 (Cartesian tensors and mathematical models of solids and viscous fluids), Semester 2, 2015\\
\vspace{8mm}
\hrule
\vspace{6mm}
 Lecture 8, 5th May2015: Navier--Stokes equations, the case of the Poiseuille flow}}
\vspace{8mm}
\hrule
\vspace{6mm}
{\Large{Pascal Grange\\
Department of mathematical sciences\\
{\ttfamily{pascal.grange@xjtlu.edu.cn}}\\
}}
\noindent\hrulefill
\end{flushleft}}
\date{}
\author{}
\maketitle
%\noindent\hrulefill
\vspace{-9mm}
 
{\bf{Keywords}}. Navier--Stokes equations, Poiseuille flow, viscosity, viscosimetry.\\
\vspace{3mm}

\tableofcontents 

\vspace{8mm}





\section{Navier--Stokes equations for incompressible fluids}

Let us remind the affine form of the material law of Newtonian fluids.
\begin{equation}
\sigma_{ij}( \vec{x}, t ) = -P(\vec{x}, t ) \delta_{ij}+ \mu\left( \frac{\partial v_i}{\partial x_j} + \frac{\partial v_j}{\partial x_i}\right).
 \label{sigmaNewton}
\end{equation}
 In the case of thin layers of fluid, this model corresponds to friction forces that are proportional to the relative velocity 
 between thin layers of fluid. For incompressible fluids, the divergence of the velocity field is zero, hence the 
 equations of motion become the Navier--Stokes equation:
 \begin{equation}
 \frac{\partial \vec{v}}{\partial t} + ( \vec{v}.\vec{\nabla}) \vec{v}= \fVol - \vec{\nabla} P + \mu \Laplace \vec{v}.
\label{NS}
 \end{equation}
In this lecture we will practice the integration of the Navier--Stokes equations,
 and propose a crucial experiment that allows one to decide whether a given fluid is Newtonian,
 and if it is to measure its viscosity.


\section{Solution of the Navier--Stokes equations for the Poiseuille flow}

The Poiseuille flow\footnote{named after the French physicist Jean Poiseuille, who first publised (in 1840) the relation 
 between the flow rate of a tube, the difference of pressure between the ends and the fourth power of the radius 
 which we will derive. The motivation was the study of the flow of blood in arteries.} is a permanent flow in a cylindrical pipe (of radius $R$, say),
which for definiteness we will assumed to be oriented vertically (in order to take gravity 
into account).\\

 As the Navier-Stokes equations are non-linear (due to the
convection term) and of order 2 in all spatial derivatives, they are extremely 
difficult to solve in full generality, so we have to make 
 additional assumptions when solving them. We will follow these steps:\\
- Propose a functional form for the velocity field, based on the symmetries of the 
 problem;\\
- Write the incompressibility condition;\\
- Write down the terms in the Navier--Stokes equations in terms of the 
 components of the velocity, one by one (using the incompressibility to simplify the expression when it is 
 possible).\\
- Write a set of three scalar PDEs by projecting each term on each of the axes 
 in the coordinate system.\\
- Solve these equations, with  boundary conditions given by the non-slipping condition, introducing as many
 intergation constants as we need.

\subsection{Form of the velocity field}
 The system has cyclindrical symmetry around the axis $\vec{e_3}$ of the 
 pipe, and we are interested in permanent flows, so we propose to look for 
 solutions in which the only non-zero component of the velocity is along $\vec{e_3}$, and does not depend on time:\\
\begin{equation}
 \boxed{
 \vec{v}( x,y,z,t ) = w( x,y,z) \vec{e_3},}
 \label{ansatz}
\end{equation}
 where $w$ is a scalar function, and $(x,y,z)$ denote the Cartesian coordinates 
 in the base $\eBase$, i.e. the position of a point in space is given by $\vec{x} = x_i \vec{e_i} = x\vec{e_1} + y \vec{e_2} + z \vec{e_3}$,
 in other words we introduce the notations $x=x_1$, $y=x_2$ and $z = x_3$.
 If we define the components of the velocity by equation $\vec{v} = v_i \vec{e_i}$, Eq.  \ref{ansatz}
 expresses that $v_1 = v_2 = 0$ in the solutions we are looking for. 

\subsection{Incompressibility}

Since the only non-zero component of the velocity field is along $\vec{e_3}$, 
the equation $\frac{\partial v_i}{\partial x_i} = 0 $ becomes
 \begin{equation}
\boxed{
\frac{\partial w}{ \partial z} = 0.}
 \label{conservation}
 \end{equation}


\subsection{Explicit form of each term in the Navier--Stokes equations}
 Let us go through the terms in Navier--Stokes equations one by one.\\

 As we are interested in permanent flows, the partial derivative w.r.t. time reduces to zero, hence
 \begin{equation}
\boxed{
\frac{\partial \vec{v}}{ \partial t} =\vec{0}.}
 \label{timeDerivative}
 \end{equation}

 In the convection term, the differential operator consists of just one
 term, $\vec{v}.\vec{\nabla} = w \frac{\partial}{\partial z}$, and when
 applied to the velocity field it returns a vector colinear to $\vec{e_3}$:
 \begin{equation}
 (\vec{v}.\vec{\nabla})\vec{v} = \left(w \frac{\partial }{\partial z} \right) (w \vec{e_3}) =  \left(w \frac{\partial w}{\partial z}\right) \vec{e_3},
 \end{equation}
but using incompressibility (Eq. \ref{conservation}), we see that the coefficient reduces to zero:\\
\begin{equation}
\boxed{
 (\vec{v}.\vec{\nabla})\vec{v} = \vec{0}.}
\label{convection}
 \end{equation}
The volume forces reduce to gravity:
\begin{equation}
\fVol = -\rho g \vec{e_3}.
\label{fVol}
\end{equation}

The pressure term have components in the three directions:
\begin{equation}
\boxed{
 -\vec{\nabla} P = - \frac{\partial P}{\partial x}\vec{e_1} -  \frac{\partial P}{\partial y}\vec{e_2}-  \frac{\partial P}{\partial z}\vec{e_3}.}
 \label{pressure}
\end{equation}

The viscosity term consists of just two terms, as the term $\frac{\partial^2 w}{\partial z^2} = \frac{\partial }{\partial }\left( \frac{\partial w}{\partial z } \right)$
 vanishes due to incompressibility (Eq. \ref{conservation}):
\begin{equation}
\boxed{
\mu \Delta \vec{v} = \mu\left(\frac{\partial^2 w}{\partial x^2} + \frac{\partial^2 w}{\partial y^2} + \frac{\partial^2 w}{\partial z^2}   \right) \vec{e_3} = \mu\left(\frac{\partial^2 w}{\partial x^2} + \frac{\partial^2 w}{\partial y^2}   \right) \vec{e_3}.}
 \label{viscosity}
\end{equation}

Collecting the terms colinear to be base vectors in $\eBase$ from Eqs. \ref{timeDerivative},\ref{convection},\ref{fVol},\ref{pressure},\ref{viscosity}, we
 ontain the following system of three scalar PDEs:
\begin{equation}
  \begin{array}{ll} &0 = -\frac{\partial P}{ \partial x} \\
   &0 = -\frac{\partial P}{ \partial y}\\
   &0 =  -\rho g - \frac{\partial P}{ \partial z} +\mu\left(\frac{\partial^2 w}{\partial x^2} + \frac{\partial^2 w}{\partial y^2}   \right)\\
   \end{array}
\label{system}
\end{equation}

\subsection{Solution of the equations}
The first two equations in the system \ref{system} imply that $P$ depends only on the coordinate $z$, hence there exists a function
 $p$ of just one variable such that 
\begin{equation}
 P(x,y,z) = p(z).
\end{equation}
Hence the last equation of the sytem can be written as:
\begin{equation}
\frac{d}{dz}( p( z ) + \rho g z ) = \mu \left(\frac{\partial^2 w}{\partial x^2} + \frac{\partial^2 w}{\partial y^2}   \right).
\label{axisEq}
\end{equation}
 The l.h.s. of Eq. \ref{axisEq} is a function of the variable $z$, but the r.h.s. depends on $x$ and $y$ only,
 since incompressibility implies $\frac{\partial w}{\partial z} = 0$. So there exists a function of two variables (denoted by $\tilde{w}$) such 
 that $w(x,y,z) = {\tilde{w}}(x,y)$ for all $x,y,z$.\\ 
Hence we can separate variables in Eq. \ref{axisEq}, bacause the l.h.s. depends only on $z$ and the $r.h.s$ depends only 
 on $x$ and $y$:
\begin{equation}
\frac{d}{dz}( p( z ) + \rho g z ) = \mu \left(\frac{\partial^2 {\tilde{w}}}{\partial x^2} + \frac{\partial^2 {\tilde{w}}}{\partial y^2}   \right)(x,y),
\label{axisEqSep}
\end{equation}
hence there exists a real constant, denoted by $C$, such that
  the two sides of Eq. \ref{axisEqSep} verify:
\begin{equation}
 C = \frac{d}{dz}( p( z ) + \rho g z ),
\label{CDef}
\end{equation}

\begin{equation}
C =  \mu \left(\frac{\partial^2 {\tilde{w}}}{\partial x^2} + \frac{\partial^2 {\tilde{w}}}{\partial y^2}   \right)(x,y).
\label{constxy}
\end{equation}

 Since we are using Cartesian coordinates (and not cylindrical coordinates), the form we proposed in Eq. \ref{ansatz}
 does not make full use of the cylindrical symmetry of the problem. In order to keep working in Cartesian coordinates (which 
 we chose to do because differential operators are simpler in Cartesian coordinates than in cylindrical coordinates), we 
 define the function $r$ of two variables by 
\begin{equation}
 r( x,y) = \sqrt{x^2 + y^2},
\end{equation}
 which measures the distance to the axis of the pipe, and we
assume there exists a function $\phi$ of just one variable such that
\begin{equation}
 w( x,y ) = \psi( r( x,y)).
\end{equation}
 In other words $w = \phi \circ r$. We can express partial derivatives of 
 the  function $w$ using the rules for the differentiation of 
 the composition of functions:\\

\begin{equation}
\frac{\partial {\tilde{w}} }{\partial x} (x,y)= \frac{\partial r }{\partial x}(x,y) \psi'( r(x,y)) = \frac{x}{\sqrt{x^2 + y^2}}\psi'( r(x,y)),
\end{equation}

\begin{equation}
\frac{\partial^2 {\tilde{w}} }{\partial x^2} (x,y)= \frac{1 }{ \sqrt{x^2 + y^2}} \psi'( r(x,y)) - \frac{x^2}{(x^2 + y^2 )^{\frac{3}{2}}}\psi'( r(x,y)) 
+  \frac{x^2}{x^2 + y^2}\psi''( r(x,y)),
\end{equation}

\begin{equation}
\frac{\partial {\tilde{w}}}{\partial y} (x,y)= \frac{\partial r }{\partial y}(x,y) \psi'( r(x,y)) = \frac{y}{\sqrt{x^2 + y^2}}\psi'( r(x,y)),
\end{equation}

\begin{equation}
\frac{\partial^2 {\tilde{w}} }{\partial y^2} (x,y)= \frac{1 }{ \sqrt{x^2 + y^2}} \psi'( r(x,y)) - \frac{y^2}{(x^2 + y^2 )^{\frac{3}{2}}}\psi'( r(x,y)) 
+  \frac{y^2}{x^2 + y^2}\psi''( r(x,y)),
\end{equation}
so that the Laplacian of ${\tilde{w}}$ reduces to

\begin{equation}
 \begin{array}{ll} 
\frac{\partial^2  {\tilde{w}}}{\partial x^2} (x,y)+ \frac{\partial^2 w }{\partial y^2} (x,y) &= \frac{2 }{ \sqrt{x^2 + y^2}} \psi'( r(x,y)) - \frac{x^2+ y^2}{(x^2 + y^2 )^{\frac{3}{2}}}\psi'( r(x,y)) + \psi''( r(x,y))\\
 &= \frac{\psi'( r(x,y) )}{\sqrt{x^2 + y^2}} +  \psi''( r(x,y)).\\
 \end{array}
\label{Laplace}
\end{equation}


Hence we can rewrite Eq. \ref{constxy} is terms of the radial variable $\rho$ of cylindrical coordinates as:
\begin{equation}
C = \mu\left( \frac{\psi'(\rho)}{\rho} + \psi''(\rho) \right).  
 \label{constrho}
\end{equation}
 Since the fluid is viscous, it sticks to the boundary of the pipe, 
 so we have the following boundary condition
\begin{equation}
 \psi( R ) = 0.
\end{equation}

 Let us look for a solution of Eq. \ref{constrho} of the form $\phi = A r^\alpha + {\mathrm{const}}$. If it exists
 we must have
\begin{equation}
\frac{C}{\mu} = A( \alpha + \alpha( \alpha - 1 ) )\rho^{\alpha-2},
\end{equation}
 hence $\alpha = 2$ because the l.h.s. does not depend on $\rho$, and 
\begin{equation}
A = \frac{C}{4\mu},
\end{equation}
from which we obtain
\begin{equation}
\psi(r) = \frac{C}{4\mu}\left( r^2 -R^2\right),
 \label{phiSol}
\end{equation} 
 in which the constant was determined using condition $\phi(R) = 0$. 
 In terms of the velocity field and Cartesian coordinates we find:
\begin{equation}
 \vec{v}( x,y,z,t) =   \frac{C}{4\mu}\left( r^2 - R^2 \right) \vec{e_3},
\end{equation}
The constant $C$ can be interpreted in two cases.\\
$\bullet$ {\bf{First case: flow under the influence of gravity.}} If the beginning and the end of the 
 pipe (at $z=0$ and $z = -L$, say), are at equilibrium with the atmosphere,
 $p (0)= p(-L) = P_{atm}$, and $C$ is obtained by integrating Eq. \ref{CDef}
 between $z=0$ and $z = -L$:\\
\begin{equation}
 \int_0^{-L} C dz = \int_0^{-L} \frac{d}{dz}(  p(z) + \rho g z ) dz,\;\;\;\; {\mathrm{i.e.}} \;\;\;\;C(-L-0) =  P_{atm} - P_{atm} + \rho g (-L-0),
\end{equation}
hence $C= \rho g $, hence inside 
 the pipe, where $r<R$, the function $\psi$  is negative, and the liquid
 follows from larger to lower values of $z$, as it should because it flows under the influence of gravity.\\
$\bullet$ {\bf{Second case: flow under the influence of a difference of pressure.}}
On the other hand, if one neglects gravity, the contribution 
 of volume forces becomes zero, and one obtains $p(z)= Cz + {\mathrm{const}}$, hence if $P$ is a decreasing function
 of $z$, one has $C<0$, and one finds   that $\phi$ is positive because of Eq. \ref{phiSol},
 hence the fluid flows from regions of larger pressure to regions of lower pressure, 
 in agreement with intuition.


\section{Application: the Poiseuille flow as a viscosimeter}

Let us compute the volume of liquid that passes through a
 section of the pipe during a unit of time (denote this quantity by $D$, whose unit is $m^3.s^{-1}$. Since the velocity 
 of the fluid is orthogonal to the section of the pipe It is just the integral of the scalar function $w$ over a disk of radius 
$r$:\\
\begin{equation}
  \begin{array}{ll} D&= \int_0^R 2\pi r w(r) dr\\
   & = -\frac{2\pi C}{4\mu} \int_0^R\left( R^2 - r^2\right) r dr\\
   & =- \frac{\pi C}{2\mu} \left( R^2 \times \frac{R^2}{2}- \frac{R^4}{4}\right)\\
   & = -\frac{C\pi}{8\mu} R^4.  
   \end{array}
\label{DCalc}
\end{equation}
Again, let us assume that the pipe is a cyclinder filled with a liquid, in equilibrium with the
atmosphere on top ($z=0$), equipped with a tap on the bottom (at $z=-L$). When one opens the tap, the fluid
 is at equilibrium with the atmosphere at the bottom, and starts flowing 
 under the influence of gravity, hence $C=\rho g$. Let us assume that a permanent flow is established,
 and let us measure put a bowl under the pipe for one second. We can measure the volume 
 of the fluid that flows into the bowl, and weigh it, which allows us 
 to determine the density of the fluid and the constant $C$.\\
 After repeating the experiment with different pipes of varying diameter $R$,
we can plot the measured volume as a function of $R^4$. If the resulting graph is 
 a piece of straight line going through the origin, we conclude that the fluid is Newtonian and
 we can measure the slope of the line 
 and deduce the value of the viscosity $\mu$.
 If it is not a straight line, we conclude that the fluid is not Newtonian. At a temperature of 25 degrees Celsius,
 the viscosity of water is about $10^{-3}$ Pa.s, the one of acetone is about $3.10^{-4}$ Pa.s, and the one of olive
 oil about $8.10^{-2}$ Pa.s. Non-Newtonian fluids include ketchup and toothpaste (which do not flow until 
 a sufficient stress is applied to them, and are called Bingham fluids), and by performing 
this experiment one can prove that  sand is not a Newtonian fluid. 
 









\end{document}






 










to-do: 
- biographical notices
- uniaxial

%\documentclass[12pt]{article}
%\documentclass[a4paper, 12pt]{scrartcl}
%\voffset=-0.9in
%\hoffset=-0.6in
\documentclass[DIV=12]{article}
\setlength{\textheight}{9.1in}
\setlength{\textwidth}{7in} 
\usepackage[margin=1.1in]{geometry}
%\usepackage[justification=justified,singlelinecheck=off]{caption}


%\cofoot{\footnotesize{\rm{Pascal Grange (August 2013)}} {\rm{\ttfamily{ pascal.grange@polytechnique.org}}}}


\let\oldbibliography\thebibliography
\renewcommand{\thebibliography}[1]{%
  \oldbibliography{#1}%
  \setlength{\itemsep}{-1pt}%
{\small
\bibliography{bibfile}}
}


\usepackage[dvips]{color}
\bibliographystyle{plain}

\usepackage{graphicx}
%\usepackage{pdfpages}
%\usepackage{multicol}
%\usepackage{pstricks,pst-grad}
%\usepackage{epsfig}
\usepackage{amsmath,esint,amssymb}

%\usepackage{subfig}
%\usepackage{tikz}
%\usepackage{verbatim}
\usepackage{amsmath}

%\usepackage{array}
%\pagestyle{scrheadings}

\newcommand*\Laplace{\mathop{}\!\mathbin\bigtriangleup}
\newcommand{\eBase}{(\vec{e_1}, \vec{e_2},\vec{e_3})}
\newcommand{\eBasePrime}{\left(\vec{e'_1}, \vec{e'_2},\vec{e'_3}\right)}
\newcommand{\maxX}{12}
\newcommand{\maxY}{12}
\newcommand{\vol}{\mathcal{V}}
\newcommand{\surf}{{\mathcal{S}}}
\newcommand{\sExt}{{{\mathcal{S}}\rightarrow{\mathrm{ext}}}}
\newcommand{\intVol}{\iiint}
\newcommand{\intSurf}{\oiint}
\newcommand{\fVol}{\vec{f^{vol}}}
\newcommand{\kg}{\mathrm{kg}}
\newcommand{\m}{\mathrm{m}}
\newcommand{\cm}{\mathrm{cm}}
\newcommand{\s}{\mathrm{s}}
\newcommand{\Pa}{\mathrm{Pa}}
\newcommand{\GPa}{\mathrm{GPa}}
\newcommand{\Tr}{\mathrm{Tr}}
\newcommand{\whatIf}{\operatornamewithlimits{=}}



\begin{document}

\title{
\noindent\hrulefill
\begin{flushleft}
{\Large \bf{XJTLU, MTH308 [Cartesian tensors and mathematical models of (elastic) solids and viscous fluids], Semester 2, 2014\\
\vspace{8mm}
\hrule
\vspace{6mm}
 Lecture 6, 7th April 2015: the Navier equations of linear elasticity}}
\vspace{8mm}
\hrule
\vspace{6mm}
{\Large{Pascal Grange\\
Department of mathematical sciences\\
{\ttfamily{pascal.grange@xjtlu.edu.cn}}\\
}}
\noindent\hrulefill
\end{flushleft}}
\date{}
\author{}
\maketitle
%\noindent\hrulefill
\vspace{-9mm}
 
{\bf{Keywords}}. Navier equation, elasticity, spherical shell.\\
\vspace{3mm}

\tableofcontents 

\vspace{8mm}

So far we introduced Hooke's  law as a material law for elastic solids,
 but we have not solved the induced PDEs for displacement fields.
 In this lecture we will write down these equations, called the Navier equations,
 and study them with boundary conditions in a spherical geometry.
 Throughout these notes, $\eBase$ denotes an orthonormal 
 base of ${\mathbf{R}}^3$, and the position $\vec{x}$ 
 of a point in space is described through associated Cartesian
 coordinates defined by $\vec{x} = x_i \vec{e_i}$.

\section{The Navier equations}

We start with the balance equations
\begin{equation}
\vec{0} = \fVol + \frac{\partial \sigma_{ij}}{ \partial x_j} \vec{e_i}.
 \label{balanceEq}
\end{equation}
and the material law relating the stress tensor to the deformation field $\vec{u} = u_i \vec{e_i}$,
 in the regime of linear elasticity, i.e. when all components of $\vec{u}$ are very small.
\begin{equation}   
 \epsilon_{ij} = \frac{1 + \nu}{E } \sigma_{ij} - \frac{\nu}{E} (\Tr \sigma ) \delta_{ij},
\end{equation}
 where $\epsilon$ denotes the linearized strain tensor:
\begin{equation}
\epsilon_{ij} = \frac{1}{2} \left(  \frac{\partial u_i}{\partial x_j} + \frac{\partial u_j}{\partial x_i} \right).
 \label{Hooke}
\end{equation}
We would like to express Eq. \ref{balanceEq} in terms of the field $\vec{u}$, so the first thing we have
 to do is to 'invert' Hooke's law in order to express $\sigma$ as a function of $\epsilon$. Since Hooke's 
 law is linear, this is easily done by taking the trace of both sides of Eq. \ref{Hooke}:\\
\begin{equation}
\Tr \epsilon = \frac{1+\nu}{E} \Tr \sigma - 3 \frac{\nu}{E} \Tr \sigma = \frac{1-2\nu}{ E } \Tr \sigma,
\end{equation}
 hence Hooke's law becomes
 \begin{equation}
  \sigma_{ij} = \frac{E}{1+\nu}\left(  \epsilon_{ij} + \frac{\nu}{ E} \Tr \sigma \delta_{ij}\right) =  \frac{E}{1+\nu} \epsilon_{ij}   + \frac{E\nu}{(1-2\nu)(1+\nu)}(\Tr \epsilon)\delta_{ij}.
\end{equation}
 It is usual to rewrite this equation in terms of the Lam\'e coefficients $\mu$ and $\lambda$ defined (note the factor of 2) as 
 \begin{equation}
  \sigma_{ij} = 2\mu \epsilon_{ij}+ \lambda (\Tr\epsilon) \delta_{ij},
 \label{material}
 \end{equation}
from which we read off 
 \begin{equation}
\boxed{
 \mu =   \frac{E}{2( 1+\nu)},\;\;\; \lambda =   \frac{E\nu}{(1-2\nu)(1+\nu)}.
}
\end{equation}
To describe the elastic properties of a solid, one can either choose Young's modulus and Posson's ratio 
 or the Lam\'e coefficients. Navier's equation are often written using the Lam\'e coefficients. Substituting the  
 form of the material law written in Eq. \ref{material}, we obtain for all $i$ in $[1..3]$:\\
\begin{equation}
0 = f^{vol}_i + \mu \Laplace u_i + (\lambda+ \mu) \frac{\partial}{\partial x_i}\left( \frac{\partial u_k}{\partial x_k} \right),
\end{equation}
 which are called the Navier equations of linear elasticity.


\section{Solution in the case of a spherical shell}
\subsection{Explicit form of the Navier equations (with spherical symmetry)}
Consider an elastic spherical shell (meaning the region  between concentric spheres of radius $R_1$ and $R_2$, with $R_1\leq R_2$, is filled with an
 elastic material). A uniform pressure $P_2$ is applied on the outside, a uniform pressure $P_1$ is applied 
 on the inside. Volume forces are neglected (or reather they have
 been compensated by the reaction of some support in the reference configuration). The 
 spherical shell contracts under the influence of the pressure. Compute the displacement fields.\\

$\bullet$ {\bf{Use spherical symmetry.}} As we have a spherical symmetry (because of the spherical geometry and the uniform pressure), 
 the displacement field will also have spherical symmetry, so we look for solutions of the form 
\begin{equation}
 \vec{u}( x_1,x_2, x_3 ) = \phi( \sqrt{x_i x_i} ) \vec{x},
\end{equation}
 where we expressed the coordinates in Cartesian form, in order to avoid having to look up 
 the expression of differential operators in spherical coordinates. All we have to do is therefor to compute the function $\phi$.\\

$\bullet$ {\bf{Work out all terms in the Navier equations in terms of the unknown function $\phi$.}} We have to solve the Navier equations
\begin{equation}
0 = \mu \Laplace u_i + (\lambda+ \mu) \frac{\partial}{\partial x_i}\left( \frac{\partial u_k}{\partial x_k} \right).
\end{equation}
 To that end, let us rewrite them as a set of three differential equations in the function $\phi$. We will 
 need the expression of the derivatives of the displacement fields:
\begin{equation}
 \frac{\partial u_i}{\partial x_j} ( x_1,x_2, x_3 ) = \delta_{ij} \phi(\sqrt{x_k x_k}  )  + x_i \frac{\partial }{\partial x_j}(  \phi( \sqrt{x_k x_k} ) ) 
=  \delta_{ij} \phi( \sqrt{x_k x_k}  )  + \frac{x_i x_j}{ \sqrt{x_k x_k} } \phi'( \sqrt{x_k x_k} ),
\label{derivative}
\end{equation}
 from which we can compute the divergence of the displacement field needed in Navier's equations:
\begin{equation}
\frac{\partial u_k}{\partial x_k}  ( x_1,x_2, x_3 ) = 3\phi(  \sqrt{x_k x_k}) + \sqrt{x_k x_k} \phi'( \sqrt{x_k x_k}).
\end{equation}
 The second term in Navier's equation is obtained by taking one more derivative:
\begin{equation}
\frac{\partial}{\partial x_i}\left(\frac{\partial u_k}{\partial x_k}  \right)( x_1,x_2, x_3 )
\end{equation}
\begin{equation}
= \frac{\partial}{\partial x_i} \left(3\phi(  \sqrt{x_k x_k}) + \sqrt{x_k x_k} \phi'( \sqrt{x_k x_k})\right) 
\end{equation}
\begin{equation}
= 3\frac{x_i }{ \sqrt{x_k x_k} } \phi'( \sqrt{x_k x_k} )+ \frac{x_i }{  \sqrt{x_k x_k }}\phi'( \sqrt{x_k x_k}) +  \sqrt{x_k x_k} \frac{x_i}{ \sqrt{x_k x_k }}\phi''( \sqrt{x_k x_k})
\end{equation}
\begin{equation}
= 4\frac{x_i }{ \sqrt{x_k x_k} } \phi'( \sqrt{x_k x_k} ) +  x_i\phi''( \sqrt{x_k x_k}).
\label{divGrad}
\end{equation}

On the other hand, we can express the Laplacian term as follows:
\begin{equation}
\Laplace u_i = \frac{\partial}{\partial x_j}\left(\frac{\partial u_i}{\partial x_j}  \right)( x_1,x_2, x_3 ) = \frac{x_i}{\sqrt{x_k x_k}} \phi'( \sqrt{x_k x_k} ) + \frac{ 3 x_i + \delta_{ij} x_j}{ \sqrt{x_k x_k} } \phi'( \sqrt{x_k x_k} ) + x_i x_j \frac{\partial}{\partial x_j}\left(  \frac{1}{\sqrt{x_k x_k}}  \phi'( \sqrt{x_k x_k})\right),
\end{equation}
but 
\begin{equation}
 x_i x_j  \frac{\partial}{\partial x_j}\left(  \frac{1}{\sqrt{x_k x_k}}  \phi'( \sqrt{x_k x_k})\right) = - x_i x_j \frac{x_j}{ (x_k x_k )^{\frac{3}{2}}} \phi'( \sqrt{x_k x_k})
 + x_i x_j   \frac{1}{\sqrt{x_k x_k}}\frac{x_j}{\sqrt{x_k x_k}}\phi''( \sqrt{x_k x_k} )
\end{equation}
\begin{equation}
=- \frac{x_i}{\sqrt{x_k x_k}}  \phi'( \sqrt{x_k x_k})+ x_i\phi''( \sqrt{x_k x_k} ).
\end{equation}
Putting all the terms together we find exactly the same form for the Laplacian term as in Eq. \ref{divGrad}:
\begin{equation}
 \Laplace u_i = 4\frac{x_i }{ \sqrt{x_k x_k} } \phi'( \sqrt{x_k x_k} ) +  x_i\phi''( \sqrt{x_k x_k}),
\end{equation}
hence the Navier equations take the form:
\begin{equation}
0 = (\lambda + 2\mu) x_i \left( 4\frac{1 }{ \sqrt{x_k x_k} } \phi'( \sqrt{x_k x_k} ) + \phi''( \sqrt{x_k x_k} ) \right),
\end{equation}
 for all $i$ in $[1..3]$. These three equations are identical (up to a multiplication by $x_i$ so we obtain the following scalar differential equation 
 for $\phi$:
\begin{equation}
0 = \frac{4}{r}\phi'(r) + \phi''( r ).
\end{equation}
We can look for solutions of the forl $\phi' = A r^{\alpha} + B$, and obtain $\alpha = -4$ and $B=0$. One more integration 
 gives rise to 

\begin{equation}
\boxed{\phi( r ) = -\frac{A}{3 r^3}+C.}
\end{equation}


\subsection{Determination of the integration constants using boundary conditions}
 We have to express the boundary conditions on the internal and external spheres using Hooke's law.
 From our computations of derivatives of displacements (Eq. \ref{derivative}) we obtain:\\
\begin{equation}
\epsilon_{ij} =   \delta_{ij} \phi( x_1,x_2, x_3 )  + \frac{x_i x_j}{ \sqrt{x_k x_k} } \phi'( \sqrt{x_k x_k} ).
\end{equation}
whose trace gives us
\begin{equation}
 \Tr(\epsilon) = 3\phi + \sqrt{x_k x_k} \phi'( \sqrt{x_k x_k} ),
\end{equation}
which we can express in sperical coordinates as
\begin{equation}
 \Tr(\epsilon) = 3\phi + r \phi'( r ) = 3C
\end{equation}
 Consider the following two points, one on the internal sphere $\vec{a} = R_1 \vec{e_1}$, and one on 
the outer sphere,  $\vec{a} = R_2 \vec{e_1}$. The outward-pointing unit normal vector to the shell
 is $-\vec{e_1}$ at  $\vec{a}$ and  $+\vec{e_1}$ at  $\vec{b}$, hence
\begin{equation}
 \sigma_{ij} n_j \vec{e_i} = +P_1 \vec{e_1}\;\;
 {\mathrm{on \;the \;inside}}, 
\end{equation}
and 
\begin{equation}
\sigma_{ij} n_j \vec{e_i} = -P_2 \vec{e_1}\;\;{\mathrm{on\;the\;outside.}} 
\end{equation}

 Using Hooke's law we therefore obtain (for components of indices $i=j=1$):
\begin{equation}
- P_1 =  2\mu\left( -2\frac{A}{3R_1^3} + C \right) + 3\lambda C,
\end{equation}
\begin{equation}
-P_2 =  2\mu \left(-2 \frac{A}{3R_2^3} + C\right) + 3\lambda C.
\end{equation}
from which one obtains the constants:
\begin{equation}
\boxed{A = \frac{3(P_1 - P_2)R_1^3 R_2^3}{4\mu(R_2^3 - R_1^3) },}
\end{equation}
\begin{equation}
\boxed{C = \frac{1}{3\lambda + 2\mu}\frac{P_1R_1^3 - P_2 R_2^3}{R_2^3 - R_1^3}.}
\end{equation}




\end{document}






 










to-do: 
- biographical notices
- uniaxial

%\documentclass[12pt]{article}
%\documentclass[a4paper, 12pt]{scrartcl}
%\voffset=-0.9in
%\hoffset=-0.6in
\documentclass[DIV=12]{article}
\setlength{\textheight}{9.1in}
\setlength{\textwidth}{7in} 
\usepackage[margin=1.1in]{geometry}
%\usepackage[justification=justified,singlelinecheck=off]{caption}


%\cofoot{\footnotesize{\rm{Pascal Grange (August 2013)}} {\rm{\ttfamily{ pascal.grange@polytechnique.org}}}}


\let\oldbibliography\thebibliography
\renewcommand{\thebibliography}[1]{%
  \oldbibliography{#1}%
  \setlength{\itemsep}{-1pt}%
{\small
\bibliography{bibfile}}
}


\usepackage[dvips]{color}
\bibliographystyle{plain}

\usepackage{graphicx}
%\usepackage{pdfpages}
%\usepackage{multicol}
%\usepackage{pstricks,pst-grad}
%\usepackage{epsfig}
\usepackage{amsmath,esint,amssymb}

%\usepackage{subfig}
%\usepackage{tikz}
%\usepackage{verbatim}
\usepackage{amsmath}

%\usepackage{array}
%\pagestyle{scrheadings}

\newcommand*\Laplace{\mathop{}\!\mathbin\bigtriangleup}
\newcommand{\eBase}{(\vec{e_1}, \vec{e_2},\vec{e_3})}
\newcommand{\eBasePrime}{\left(\vec{e'_1}, \vec{e'_2},\vec{e'_3}\right)}
\newcommand{\maxX}{12}
\newcommand{\maxY}{12}
\newcommand{\vol}{\mathcal{V}}
\newcommand{\surf}{{\mathcal{S}}}
\newcommand{\sExt}{{{\mathcal{S}}\rightarrow{\mathrm{ext}}}}
\newcommand{\intVol}{\iiint}
\newcommand{\intSurf}{\oiint}
\newcommand{\fVol}{\vec{f^{vol}}}
\newcommand{\kg}{\mathrm{kg}}
\newcommand{\m}{\mathrm{m}}
\newcommand{\cm}{\mathrm{cm}}
\newcommand{\s}{\mathrm{s}}
\newcommand{\Pa}{\mathrm{Pa}}
\newcommand{\GPa}{\mathrm{GPa}}
\newcommand{\Tr}{\mathrm{Tr}}
\newcommand{\whatIf}{\operatornamewithlimits{=}}
\newcommand{\etheta}{\vec{e_\theta}}
\newcommand{\er}{\vec{e_r}}
\newcommand{\ez}{\vec{e_z}}
\newcommand{\rCart}{\sqrt{x_1^2+x_2^2}}

\begin{document}

\title{
\noindent\hrulefill
\begin{flushleft}
{\Large \bf{XJTLU, MTH308 (Cartesian tensors and mathematical models of solids and viscous fluids), Semester 2, 2015\\
\vspace{8mm}
\hrule
\vspace{6mm}
 Lecture 9, 12th May, 2015: Flow between two cylinders (the Couette flow)}}
\vspace{8mm}
\hrule
\vspace{6mm}
{\Large{Pascal Grange\\
Department of mathematical sciences\\
{\ttfamily{pascal.grange@xjtlu.edu.cn}}\\
}}
\noindent\hrulefill
\end{flushleft}}
\date{}
\author{}
\maketitle
%\noindent\hrulefill
\vspace{-9mm}
 
{\bf{Keywords}}. Navier--Stokes equation, Couette flow, cylindrical coordinates.\\
\vspace{3mm}

\tableofcontents 

\vspace{8mm}





\section{The Navier--Stokes equations for the Couette flow}

Let us remind the affine form of the material law of Newtonian fluids.
\begin{equation}
\sigma_{ij}( \vec{x}, t ) = -P(\vec{x}, t ) \delta_{ij}+ \mu\left( \frac{\partial v_i}{\partial x_j} + \frac{\partial v_j}{\partial x_i}\right).
 \label{sigmaNewton}
\end{equation}
 In the case of thin layers of fluid, this model corresponds to friction forces that are proportional to the relative velocity 
 between thin layers of fluid. For incompressible fluids, the divergence of the velocity field is zero, hence the 
 equations of motion become the Navier--Stokes equation:
 \begin{equation}
 \rho\left( \frac{\partial \vec{v}}{\partial t} + ( \vec{v}.\vec{\nabla}) \vec{v}\right)= \fVol - \vec{\nabla} P + \mu \Laplace \vec{v}.
\label{NS}
 \end{equation}
In this lecture we will encounter one more example of solution of the Navier--Stokes 
 equations with cylindrical symmetry, but this time the direction of the velocity field (and not only the 
 norm) varies from point to point.




\section{The Couette flow}

The Couette flow\footnote{named after the French physicist Maurice Couette (1858-1943), who built the first viscosimeter.} is a permanent flow of an incompressible 
 fluid  occupying the space between two coaxial cylinders of radii $a$ and $b$, with $a < b$. The inner
 cylinder is fixed, while the external cylinder is rotating at constant speed  $\omega$ radians per second. As a viscous fluid adhers
 to boundaries, the velocity field is zero on the inner cylinder, while it equals the velocity of the 
 external cylinder on $r=b$.\\


Let us look for a velocity field that take the following form in  cylindrical coordinates:
 \begin{equation}
 \vec{v}( r, \theta) = v( r ) \etheta,
 \label{velocityForm}
 \end{equation}
 where $v$ is a scalar function we will have to determine, 
which means that the fluid flows in the orthoradial direction, and 
 the norm of the velocity respects the cylindrical symmetry (it depends only on the 
 distance from the axis). We also assume that the length of the cylinder is large compared to $a$ 
 and $b$, so we did not include any dependence on $z$ in the form \label{velocityForm} (this 
 is equivalent to considering an "infinite cylinder"). Moreover there is no explicit dependence in time in Eq. \ref{velocityForm},
 which means that we are interested in a steady flow.\\


\section{Solution of the equations}

\subsection{Cylindrical coordinates (see tutorial for detailed derivations)}



 For a scalar function of a point in ${\mathbf{R}}^3$ described by cylindrical coordinates $f: (r,\theta,z) \mapsto f(r,\theta,z)$, the gradient in cylindrical coordinates 
 is expressed as 
 \begin{equation}
 \boxed{
 \vec{\nabla} f = \frac{\partial f}{\partial r} \er + \frac{1}{r} \frac{\partial f}{\partial \theta} \etheta +   \frac{\partial f}{\partial z} \ez.}
 \label{gradientCyl}
 \end{equation}

We can compute the Laplacian of $f$ in cylindrical coordinates by applying  Stokes' theorem to the 
 flux of $\vec{\nabla} f$ through the boundary of an elementary volume oriented towards the exterior.
 Consider a vector field $\vec{u}= u_r \er + u_\theta \etheta + u_z \ez$ and apply Stokes' theorem to the flux of $\vec{u}$
 through an elementary volume centered at point $\vec{x}$ and spanned by the three vectors $dr \er, rd\theta \etheta, dz \ez$ (see tutorial
 for details):
\begin{equation}
dV = rd\theta dr dz,
\end{equation}
\begin{equation}
  \begin{array}{ll}  
{\mathrm{div}} \vec{u} (r,\theta, z )dV &= (u_z(r,\theta, z + dz) - u_z( r,\theta,z) )rd\theta dr\\
 & + ( (r+dr)u_r(r+dr, \theta,z) - ru_r(r,\theta,z))d\theta dz \\
 &+ ( u_\theta( r,\theta + d\theta, z) - u_\theta( r,\theta, z ) )dr dz.\\
 &= r dr d\theta dz\left(  \frac{1}{r}\frac{\partial( r u_r)}{\partial r} + \frac{1}{r} \frac{\partial u_\theta}{\partial \theta} + \frac{\partial u_z}{\partial z}\right),
 \end{array}
\label{divEquation}
\end{equation}
hence 
\begin{equation}
\boxed{
 {\mathrm{div}} \vec{u}  = \frac{1}{r}\frac{\partial}{\partial r}\left( ru_r\right)+ \frac{1}{r}\frac{\partial u_\theta}{\partial \theta}  + \frac{\partial  u_z}{\partial z}}
\end{equation}
Substituting the gradient of $f$ (Eq. \ref{gradientCyl}) to $\vec{u}$, we obtain:
\begin{equation}
\boxed{
 \Laplace f  (r,\theta, z )= \frac{1}{r}\frac{\partial}{\partial r}\left( r\frac{\partial f}{\partial r}\right)+ \frac{1}{r}\frac{\partial}{\partial \theta} \left( \frac{1}{r} 
\frac{\partial f}{\partial \theta}\right) + \frac{\partial ^2 f}{\partial z^2}.}
\label{LaplacianCyl}
\end{equation}


\subsection{Rewriting the Navier--Stokes equations in cylindrical coordinates}
The time derivative of the velocity field is zero, as usual for permanent flows: 
\begin{equation}
\boxed{
\frac{\partial \vec{v}}{ \partial t} =\vec{0}.}
 \label{timeDerivative}
 \end{equation}
Consider the expression proposed in Eq. \ref{velocityForm} for the velocity field in cylindrical 
coordinates. As usual we write down the differential operator used to compute the convection term, 
 which from Eq. \ref{gradientCyl}:
\begin{equation}
\vec{v}.\vec{\nabla} =  \frac{v( r)}{r} \frac{\partial }{ \partial \theta }. 
\end{equation}
We compute the convection term by acting with this differential on the velocity field,
 without forgetting that $\etheta$ is a not constant vector but depends on $\theta$:
\begin{equation}
(\vec{v}.\vec{\nabla} ) \vec{v} =  \frac{v( r)}{r} \frac{\partial }{ \partial \theta }( v(r) \etheta) = \frac{v( r)^2}{r} \frac{\partial }{ \partial \theta } \er
= -\frac{v( r)^2}{r} \er.
\end{equation}

Of course we obtain the same result if we use the base $\eBase$ to express the velocity field,
 because the vectors $\vec{e_1}$ and $\vec{e_2}$ are constant:
 \begin{equation}
\boxed{
(\vec{v}.\vec{\nabla} ) \vec{v} =  \frac{v( r)}{r} \frac{\partial }{ \partial \theta }\left(v(r)( -\sin\theta \vec{e_x}+ \cos\theta \vec{e_y})\right) = \frac{v( r)^2}{r} \left(-\cos\theta \vec{e_x}-\sin\theta \vec{e_y}\right) = -\frac{v(r)^2}{r}\er.}
 \end{equation}
The pressure term is a simple application of the differential operator to the scalar function $P$:
\begin{equation}
 \boxed{
 -\vec{\nabla} P = -\frac{\partial P}{\partial r} \er - \frac{1}{r} \frac{\partial P}{\partial \theta} \etheta -  \frac{\partial P}{\partial z} \ez.
 }
\end{equation}

 The Laplacian term can be computed in the same way, acting with the differential operator 
 on the vector $\vec{v}$, not forgetting to differentiate the vectors of the base:
 \begin{equation}
  \begin{array}{ll}  
 \Laplace \vec{v}(r,\theta) &=  \frac{1}{r}\frac{\partial}{\partial r}\left( r\frac{\partial (v(r) \etheta)}{\partial r}\right) + \frac{1}{r}\frac{\partial}{\partial \theta} \left( \frac{1}{r} 
\frac{\partial (v(r) \etheta)}{\partial \theta}\right)\\
 &= \left(  \frac{1}{r} \frac{d}{dr}( r v'(r)) \right) \etheta + \frac{1}{r}\frac{\partial}{\partial \theta} \left( - \frac{v(r) }{r} \er \right)\\
 & =  \left( \frac{1}{r} \frac{d}{dr}( r v'(r))  - \frac{v(r)}{r^2}\right) \etheta
 \end{array} 
 \end{equation}
Collecting the scalar coefficients of the three vectors $\er,\etheta,\ez$, we obtain the following system of equations:
\begin{equation}
\boxed{\begin{cases}
  \begin{array}{ll} 
   & -\rho\frac{v^2}{r} = -\frac{\partial P}{ \partial r}\\
   &0 = -\frac{1}{r}\frac{\partial P}{ \partial \theta} + \frac{\mu}{r}\frac{d}{dr}( r v'(r)) -\mu\frac{v(r)}{r^2}\\
   &0 =  -\rho g - \frac{\partial P}{\partial z}\\
   \end{array}\end{cases}}
\label{systemCyl}
\end{equation}

 \subsection{Integration of the Navier--Stokes equations}
 The third equation of the system is the same as in Cartesian coordinates and just expresses the hydrostatic dependence of the pressure 
 field. Integrating the second equation (the one corresponding to the coefficients along $\etheta$) is enough to determine the 
 velocity field. First of all, we notice that $\partial P / \partial \theta$ depends only on $r$:
\begin{equation}
\frac{\partial P}{ \partial \theta}  = \mu\frac{d}{dr}( r v'(r)) -\mu\frac{v(r)}{r},
 \label{eqSufficient}
\end{equation}
 hence there exists two functions of $r$ only, call them $C$ and $D$, such that
\begin{equation}
 P(r,\theta, z ) = C( r ) \theta -\rho g z + D(r ).
\end{equation}
 However, since $\theta$ is defined up to a multiple of $2\pi$, the function $\theta \mapsto P(r,\theta,z)$ must be
 $2\pi$-periodic for all values of $r$ and $z$ between the two cylinders. The only way to 
 satisfy this periodicity condition is to have $C(r) = 0$. Hence the pressure does not depend on the orthoradial
 angle:
\begin{equation}
\frac{\partial P}{\partial \theta} = 0,
\end{equation}
 and Eq. \ref{eqSufficient} become an equation in $v$ only:\\
\begin{equation}
\frac{d}{dr}( r v'(r)) - \frac{v(r)}{r}= 0.
\end{equation}
We rewrite it as
\begin{equation}
v'' + \frac{v'}{r}- \frac{v}{r^2} = 0.
\end{equation}
We integrate this equation once: there exists a constant $D$ such that
\begin{equation}
 v' +\frac{v}{r} = D.
\end{equation}
 The function $r\mapsto Dr/2$ is a particular solution of this equation, and we
 can introduce another constant $E$ (and redefine the unknown constant $D$) such that
\begin{equation}
 v( r ) = \frac{E}{r} + Dr.
\end{equation}
 Since the radius of the inner cylinder is strictly positive, this expression is finite
 for all values of $r$ corresponding to points between the two cylinders, 
 and we can determine the two constants using the two boundary conditions:
\begin{equation}
 v( a ) = 0,\;\;\; v( b ) = b\omega.
\end{equation}
 Hence
\begin{equation}
 \begin{cases}
  \begin{array}{ll}  
  0 & = \frac{E}{a} + Da \\
  b\omega & = \frac{E}{b} + Db.
 \end{array}
\end{cases} 
 \end{equation}
Hence
 \begin{equation}
 D = \frac{b^2\omega}{b^2 - a^2},\;\;\; E = -\frac{a^2b^2\omega}{ b^2 - a^2}.
 \end{equation}
 from which we obtain the expression of the velocity field:
\begin{equation}
\boxed{
\vec{v}(r,\theta,z) = \left( \frac{b^2\omega}{ b^2 - a^2}\left( -\frac{a^2}{r} + r \right) \right) \etheta.}
\end{equation}
 and also the pressure by integration of the component of the Navier--Stokes equation along direction $\er$:
\begin{equation}
P(r,\theta,z) = -\rho g z + \rho\int_a^b\left( \frac{v^2(r)}{r}\right)dr + constant
\end{equation}



\end{document}

\section{Application: the Couette flow as a viscosimeter}
 In order to maintain the movement of the external cylinder at a constant
 speed, the operator must exert a force to compensate the friction 
 force exerted by the fluid on the external cylinder.\\

This force must be orthoradial, and to compute it we have to use the expression 
 of the stress tensor, which we have in Cartesian coordinates. But by cylindrical 
 symmetry, it is enough to compute the norm of the force at point of coordinates 
  $(x_1 = b, x_2 = 0, z = 0)$, at which $r=b$ and $\etheta = \vec{e_2}$.
 \begin{equation}
\vec{v}(\vec{x}) = u_1 \vec{e_1} + u_1 \vec{e_2},
\end{equation}
where from the expression in cylindrical coordinates we have
 \begin{equation}
u_1 =  -\frac{x_2}{\sqrt{x_1^2+x_2^2}}\left(\frac{E}{\sqrt{x_1^2+x_2^2}} +  D \sqrt{x_1^2+x_2^2}\right)
\end{equation}
\begin{equation}
u_1 =  \frac{x_1}{\sqrt{x_1^2+x_2^2}}\left(\frac{E}{\sqrt{x_1^2+x_2^2}} +  D \sqrt{x_1^2+x_2^2}\right)
\end{equation}






 \subsection{Incompressibility}

Since the only non-zero component of the velocity field are 
 along vectors $\vec{e_1}$ and $\vec{e_2}$, we can write the incompressibility condition as:
 \begin{equation}
\frac{\partial u_1}{\partial x_1} + \frac{\partial u_2}{\partial x_2} = 0.
 \label{conservation}
 \end{equation}
Let us compute the two terms in the sum explicitly, as we will need them in the Navier--Stokes equations:
\begin{equation}
\frac{\partial u_1}{\partial x_1} = 
\end{equation}

\begin{equation}
\frac{\partial u_2}{\partial x_2} = 
\end{equation}

\subsection{Explicit form of each term in the Navier--Stokes equations}
 Let us go through the terms in Navier--Stokes equations one by one.\\

 As we are interested in permanent flows, the partial derivative w.r.t. time reduces to zero, hence
 \begin{equation}
\boxed{
\frac{\partial \vec{v}}{ \partial t} =\vec{0}.}
 \label{timeDerivative}
 \end{equation}

 In the convection term, the differential operator consists of just one
 term, $\vec{v}.\vec{nabla} = w \frac{\partial}{\partial z}$, and when
 applied to the velocity field it returns a vector colinear to $\vec{e_3}$:
 \begin{equation}
 (\vec{v}.\vec{\nabla})\vec{v} = \left(w \frac{\partial }{\partial z} \right) (w \vec{e_3}) =  \left(w \frac{\partial w}{\partial z}\right) \vec{e_3},
 \end{equation}
but using incompressibility (Eq. \ref{conservation}), we see that the coefficient reduces to zero:\\
\begin{equation}
\boxed{
 (\vec{v}.\vec{\nabla})\vec{v} = \vec{0}.}
\label{convection}
 \end{equation}
The volume forces reduce to gravity:
\begin{equation}
\fVol = -\rho g \vec{e_3}.
\label{fVol}
\end{equation}

The pressure term have components in the three directions:
\begin{equation}
\boxed{
 -\vec{\nabla} P = - \frac{\partial P}{\partial x}\vec{e_1} -  \frac{\partial P}{\partial y}\vec{e_2}-  \frac{\partial P}{\partial z}\vec{e_3}.}
 \label{pressureGradient}
\end{equation}

The viscosity term consists of just two terms, as the term $\frac{\partial^2 w}{\partial z^2} = \frac{\partial }{\partial }\left( \frac{\partial w}{\partial z } \right)$
 vanishes due to incompressibility (Eq. \ref{conservation}):
\begin{equation}
\boxed{
\mu \Delta \vec{v} = \mu\left(\frac{\partial^2 w}{\partial x^2} + \frac{\partial^2 w}{\partial y^2} + \frac{\partial^2 w}{\partial z^2}   \right) \vec{e_3} = \mu\left(\frac{\partial^2 w}{\partial x^2} + \frac{\partial^2 w}{\partial y^2}   \right) \vec{e_3}.}
 \label{viscosity}
\end{equation}

Collecting the terms colinear to be base vectors in $\eBase$ from Eqs. \ref{timeDerivative},\ref{convection},\ref{fVol},\ref{pressure},\ref{viscosity}, we
 ontain the following system of three scalar PDEs:
\begin{equation}
  \begin{array}{ll} &0 = -\frac{\partial P}{ \partial x} \\
   &0 = -\frac{\partial P}{ \partial y}\\
   &0 =  -\rho g - \frac{\partial P}{ \partial z} +\mu\left(\frac{\partial^2 w}{\partial x^2} + \frac{\partial^2 w}{\partial y^2}   \right)\\
   \end{array}
\label{system}
\end{equation}










to-do: 
- biographical notices
- uniaxial

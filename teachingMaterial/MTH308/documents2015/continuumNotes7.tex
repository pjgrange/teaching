%\documentclass[12pt]{article}
%\documentclass[a4paper, 12pt]{scrartcl}
%\voffset=-0.9in
%\hoffset=-0.6in
\documentclass[DIV=12]{article}
\setlength{\textheight}{9.1in}
\setlength{\textwidth}{7in} 
\usepackage[margin=1.1in]{geometry}
%\usepackage[justification=justified,singlelinecheck=off]{caption}


%\cofoot{\footnotesize{\rm{Pascal Grange (August 2013)}} {\rm{\ttfamily{ pascal.grange@polytechnique.org}}}}


\let\oldbibliography\thebibliography
\renewcommand{\thebibliography}[1]{%
  \oldbibliography{#1}%
  \setlength{\itemsep}{-1pt}%
{\small
\bibliography{bibfile}}
}


\usepackage[dvips]{color}
\bibliographystyle{plain}

\usepackage{graphicx}
%\usepackage{pdfpages}
%\usepackage{multicol}
%\usepackage{pstricks,pst-grad}
%\usepackage{epsfig}
\usepackage{amsmath,esint,amssymb}

%\usepackage{subfig}
%\usepackage{tikz}
%\usepackage{verbatim}
\usepackage{amsmath}

%\usepackage{array}
%\pagestyle{scrheadings}

\newcommand*\Laplace{\mathop{}\!\mathbin\bigtriangleup}
\newcommand{\eBase}{(\vec{e_1}, \vec{e_2},\vec{e_3})}
\newcommand{\eBasePrime}{\left(\vec{e'_1}, \vec{e'_2},\vec{e'_3}\right)}
\newcommand{\maxX}{12}
\newcommand{\maxY}{12}
\newcommand{\vol}{\mathcal{V}}
\newcommand{\surf}{{\mathcal{S}}}
\newcommand{\sExt}{{{\mathcal{S}}\rightarrow{\mathrm{ext}}}}
\newcommand{\intVol}{\iiint}
\newcommand{\intSurf}{\oiint}
\newcommand{\fVol}{\vec{f^{vol}}}
\newcommand{\kg}{\mathrm{kg}}
\newcommand{\m}{\mathrm{m}}
\newcommand{\cm}{\mathrm{cm}}
\newcommand{\s}{\mathrm{s}}
\newcommand{\Pa}{\mathrm{Pa}}
\newcommand{\GPa}{\mathrm{GPa}}
\newcommand{\Tr}{\mathrm{Tr}}
\newcommand{\whatIf}{\operatornamewithlimits{=}}



\begin{document}

\title{
\noindent\hrulefill
\begin{flushleft}
{\Large \bf{XJTLU, MTH308 (Cartesian tensors and mathematical models of [elastic] solids and viscous fluids), Semester 2, 2015\\
\vspace{8mm}
\hrule
\vspace{6mm}
 Lecture 7, 28th April 2015: Newtonian fluids, and the Navier--Stokes equations for incompressible fluids}}
\vspace{8mm}
\hrule
\vspace{6mm}
{\Large{Pascal Grange\\
Department of mathematical sciences\\
{\ttfamily{pascal.grange@xjtlu.edu.cn}}\\
}}
\noindent\hrulefill
\end{flushleft}}
\date{}
\author{}
\maketitle
%\noindent\hrulefill
\vspace{-9mm}
 
{\bf{Keywords}}. Description of fluids, conservation of mass, Material laws, Newtonian fluids.\\
\vspace{3mm}

\tableofcontents

\vspace{8mm}


\section{Summary of the module so far}

 We have developed a general framework to study  continuous media.\\
 $(i)$ Continuum assumption: distances are large in scale of the size of molecules.\\
 $(ii)$ Kinematics: Lagrangian description based on trajectories $\vec{\Phi}( \vec{X}, t )$, Eulerian description based on a velocity field $\vec{v}(\vec{x},t)$. The 
 two descriptions are related by $\vec{v}( \vec{\Phi}(\vec{X},t),t) = \frac{\partial}{\partial t} \vec{\Phi}(\vec{X},t)$.\\
 $(iii)$ bulk forces and surface forces. Cauchy's assumption: consider a sample of continuous medium
 in a volume $\mathcal{V}$, whose boundary $\surf$ is oriented towards the exterior of $\mathcal{V}$, this
 sample feels the action of the rest of the continuous medium through the force $\intSurf \sigma_{ij}( \vec{x}) n_j( \vec{x}) dS$,
 where $\sigma$ is the stress tensor and $\vec{n}(\vec{x}) = n_j(\vec{x}) \vec{e_j}$ is the normal vector to $\surf$ at point $\vec{x}$, and
 $\eBase$ is an orthonormal base.\\

We specialized this framework for the study of elastic solids (in the regime of small deformations) and we have started doing it for 
 fluids.\\

$\bullet$ {\bf{Elastic solids.}}\\
 $(i)$ We study equilibrium states. Thanks to Stoke's theorem the balance laws can be written as follows for all $i$ in $[1..3]$:\\
 \begin{equation}
\vec{0} = \fVol + \frac{\partial \sigma_{ij}}{\partial x_j} \vec{e_i}
 \end{equation}
 $(ii)$ If forces are small enough, deformations are small and reversible. Computations are done at first order in the 
 deformation field $\vec{u} = \Phi( \vec{X}, t ) - \vec{X}$.\\
 $(iii)$ The stress tensor in terms of the linearized strain tensor $\epsilon$ by Hooke's law,
 expressed using two parameters, Young's modulus $E$ (order of magnitude: a few hundreds of gigapascals for steel, about 10 gigapascals for wood, about $0.1$ gigapascal for rubber),
 and Poisson's ratio $\nu$ (a positive number which can be shown to be smaller than $0.5$):\\
 \begin{equation}
 \epsilon_{ij} = \frac{1+\nu}{E}\sigma_{ij} -\frac{\nu}{E}(\Tr\sigma)\delta_{ij}\;\;\;\;\; \epsilon_{ij} = \frac{1}{2}\left( \frac{\partial u_i}{\partial x_j}+\frac{\partial u_j}{\partial x_i} \right) .
\end{equation}

$\bullet$ {\bf{Fluids.}}\\
 $(i)$ We study flows, deformations are large, hence we need to include the acceleration terms in the equations of motion:\\
 \begin{equation}
\rho\left( \frac{\partial \vec{v}}{\partial t} + ( \vec{v}.\vec{\nabla}) \vec{v} \right)= \fVol + \frac{\partial \sigma_{ij}}{\partial x_j} \vec{e_i}.
 \label{eomFluids}
 \end{equation}
$(ii)$ We need a {\emph{material law}} (analogous to Hooke's law) to express the stress tensor $\sigma$ in terms of the 
 other parameters of the problem.\\


\section{Reminders on fluids}
 Fluids are continuous media that can undergo large deformation
 without losing their continuity property: for example a volume of water can be 
transferred between two containers of different shapes, and it will still be
 a single volume of water (provided the pressure and temperature conditions are sufficiently
 similar in the two containers), whereas for elastic solids the identity 
 of the material can be modified even by relatively small deformation (for a bar of steel,
 the linear regime of elasticity is only valid for $\delta L /L \simeq 5.10^{-3}$, 
 and breakage, i.e. loss of continuity, can occur if $\delta L/L$ reaches a few percent).\\

This will have several consequences for us:\\
1. We will need a description of speed and position that easily accommodates $\vec{\Phi}(\vec{X}, t ) = \vec{X} + \vec{u}( \vec{X}, t )$
where $|| \vec{u }||$ is not necesseraly small in scale of $||\vec{X}||$; this description is called the Euler description;\\
2. As we will study flows, not only hydrostatics, we will need to add an acceleration term to the equations of motion.\\   


\subsection{The Euler description of fluids}

The Euler velocity field is a vector field (it associates 
 a vector to every point in space, at every time), denoted by $\vec{v}( \vec{x}, t )$.
Its value is the speed of the fluid particle that is at point $\vec{x}$ at time $t$. The relation to 
 the Lagrange description is the following: the particle that is at point $\vec{x}$ at time $t$
 was at some point $\vec{X}$ at time $0$, so for this particle one has $\vec{x} = \vec{\Phi}( \vec{x}, t  )$.
 its speed at time $t$ is just the derivative of the position $\vec{\Phi}( \vec{x}, t  )$ with respect to time, hence
 one can compute the Euler velocity field in terms of the Lagrange flow function $\vec{\Phi}$ as follows:
\[
\boxed{
\vec{v}( \vec{\Phi}(\vec{X}, t),t) = \frac{\partial}{\partial t}\vec{\Phi}(\vec{X}, t ).
}
\label{equivalence}
\]
In the case of the flow of a river, or inside a pipe, we will write the equations
of motion in terms of the Euler velocity field $\vec{v}$ and its derivative, 
 as it is more practical to measure the velocity of the fluid that is passing 
 in front of the observer (at point $\vec{x}$) than to ask oneself where each particle was
 at time $0$.\\


{\bf{Remark ('What is the Euler description for elastic solids ?').}} In the case of elastic solids
 we consider small deformations. We computed all the effects at the lowest non-zero orders
 in the deformation (and their derivatives) and only used the Lagrange description. Let us see 
 what the relation between the two descriptions becomes for small deformation:\\
\begin{equation}
\vec{\Phi}(\vec{X},t) = \vec{X} + \vec{u}(\vec{X},t),
\end{equation}
 hence the relation between Euler and Lagrange descriptions becomes
\begin{equation}
\vec{v}(  \vec{X} + \vec{u}(\vec{X},t) ,t) =  \frac{\partial}{\partial t}\vec{u}(\vec{X},t),
\end{equation}
 from which we see that the velocity is of order one in $u$, hence if we are working at 
lowest non-zero order in $u$ (and its derivatives), we can replace the argument  $\vec{X} + \vec{u}$ by $u$,
 and the Eulerian field is just the time derivative of the Lagrange flow, expressed at the same point:
\begin{equation}
\vec{v}(  \vec{X},t) =  \frac{\partial}{\partial t}\vec{u}(\vec{X},t) + o(\vec{u}),
\label{solidCase}
\end{equation}
 hence the Lagrange and Euler descriptions are 'trivially' equivalent for small deformations.\\

The Euler description will be used to write the equation 
 of motions for fluids. However, the Lagrange viewpoint (in which one follows the 
 same particles of fluids along their trajectory) 
 can be useful when establishing  the equations, as we are going 
 to see in the next two sections.

\subsection{Conservation of mass ("the continuity equation")}

Consider the fluid enclosed in volume $\mathcal{V}$ at time $t$. We can write 
 its mass $M$ as 
\begin{equation}
M = \iiint_{\mathcal{V}} \rho( \vec{x}, t ) dV.
\end{equation}
 where $\rho$ is the density of the fluid, which {\emph{a priori}} depends on both
 space and time. If we consider the same particles of fluid at time $t+\epsilon$ (for some 
 small time $\epsilon$), its mass will still be $M$ by conservation of matter,
 but its mathematical expression (including terms of order up to one in $\epsilon$) will consist of two terms:
 one is the volume integral of the density $\rho(\vec{x}, t+\epsilon$) over volume $V$,
 and the other one is the surface integral corresponding to the mass particles of fluids
 that went through the surface between time $t$ and time $t+\epsilon$ (if a particle goes through the
 boundary $\mathcal{S}$ at point $\vec{x}$, with the outgoing unit normal vector $\vec{n}( \vec{x})$ at this point, 
 the scalar product $\vec{v}(\vec{x}, t).\vec{n}(\vec{x})$ is positive if the particle is leaving the volume $\mathcal{V}$, and we must 
 add its mass $\vec{v}(\vec{x}, t).\vec{n}(\vec{x}) \epsilon dS$ to find $M$; if the dot-product is negative, the 
  term $\vec{v}(\vec{x}, t).\vec{n}(\vec{x}) \epsilon dS$ is negative and equals the opposite  mass of a fluid particle that was not in volume
 $\mathcal{V}$ at time $t$ but came in between $t$ and $t+\epsilon$). Hence we write the new expression of $M$ at time $t+\epsilon$ as 
\[
\begin{array}{lll}
M &= & \iiint_{\mathcal{V}} \rho( \vec{x}, t+\epsilon ) dV + \oiint_\sExt \rho (\vec{x}, t)\vec{v}(\vec{x}, t).\vec{n}(\vec{x}) \epsilon dS + o(\epsilon)\\
    &  = & \iiint_{\mathcal{V}} \rho( \vec{x}, t) dV + \iiint_{\mathcal{V}} \frac{\partial\rho}{\partial t}( \vec{x}, t) \epsilon dV +
                 \epsilon\iiint_{\mathcal{V}}\frac{\partial (\rho\vec{v})}{\partial x_j}(\vec{x},t) dV + o(\epsilon)\\
   & = & M +  \epsilon\left( \iiint_{\mathcal{V}}\left( \frac{\partial\rho}{\partial t}( \vec{x}, t)+ \frac{\partial (\rho\vec{v})}{\partial x_j}(\vec{x},t)\right) dV\right) + o(\epsilon)
 \end{array}
\label{massAfter}
\]
where we used Stoke's theorem in the second equality. We can write the following equality, which expresses that the derivative of the 
 mass $m$ with respect to $\epsilon$ is zero:  
\begin{equation}
\iiint_{\mathcal{V}}\left( \frac{\partial\rho}{\partial t}( \vec{x}, t)+ \frac{\partial (\rho\vec{v})}{\partial x_j}(\vec{x},t)\right) dV = 0.
\label{massConsGlobal}
\end{equation}
 Since Eq. \ref{massConsGlobal} holds for any volume
 $\mathcal{V}$, it must hold locally (at scales where the continuous description holds), hence
 the conservation of mass is express as the following PDE in the density and velocity fields:
\begin{equation}
\boxed{
\frac{\partial\rho}{\partial t}( \vec{x}, t)+ \frac{\partial (\rho v_j)}{\partial x_j}( \vec{x},t) = 0.}
\label{massCons}
\end{equation}
 This equation is sometimes called the continuity equation (but the mathematical assumptions
 on functions are stronger than continuity, as we assume that derivatives exist; in this context,
 {\emph{continuity}} refers to {\emph{conservation}} of a quantity, and express that {\emph{the amount
 of a conserved quantity that enters
 a domain equals the amount that is stored}}). \\

{\bf{Example (incompressible fluid).}} In the case where one assumes that the density 
 is uniform and constant (i.e. that the fluid is incompressible), one can write $\rho(\vec{x},t) = \rho_0$,
 for example $\rho_0 = 10^3 \kg .\m^{-3}$ for water, then Eq. \ref{massCons} becomes and 
 equation in the velocity field only:
\begin{equation}
 \frac{\partial v_j}{\partial x_j} = 0.
 \label{incomp}
\end{equation}
 which is often written as ${\mathrm{div}} \vec{v} = 0$, introducing the 
 divergence operator. Hence, given an expression 
 for the velocity fluid of a fluid, one can decide if the fluid is incompressible
 just by computing the divergence of the velocity field.




\subsection{Acceleration of a particle of fluid}


So far we have studied elasticity problems and wrote that the sum of volume  forces
 and surface forces equals zero. The resulting equation applies to static problems. When
 we want to write down the dynamics of a flow, the sum of forces 
 takes the same form, but it equals the variation rate of the impulsion.
 We therefore have to compute the variation of the impulsion of 
 the impulsion of the quantity of fluid contained in volume $dV$ at time $t$, that is:
\begin{equation}
 \vec{p} = \rho(\vec{x},t) dV \vec{v}(\vec{x},t),
 \label{impulsion}
\end{equation}
 between time $t$ and time $t+\epsilon$, at first order in $\epsilon$.
 The quantity $\vec{p}$ in Eq. \ref{impulsion} is expressed in Eulerian variables,
 but we are following the same particle of fluid on its trajectory between times
 $t$ and $t+\epsilon$ (which is a Lagrangian approach to the problem, but is still tractable 
 if $\epsilon$ is small enough). First of all, the mass of the particle $\vec{\rho}(\vec{x},t) dV$
 is conserved between times $t$ and $t+\epsilon$ (the density may vary, but its variation is compensated 
by a variation of the volume element $dV$), so all we have to do is to compute the 
variation $\delta\vec{v}$ of the velocity of the particle, at first order in $\epsilon$:
\begin{equation}
\delta \vec{v} = \vec{v}( \vec{x} + \vec{v}( \vec{x}, t ) \epsilon + o(\epsilon) ,t + \epsilon) - \vec{v}( \vec{x},t). 
\label{DLVitesse} 
\end{equation}
 The delicate point comes from the fact that {\emph{both}} position and time vary during the move. We can expand 
 the first term in Eq. \ref{DLVitesse} at first order in $\epsilon$, and both time and space derivatives of the
Eulerian velocity appear:
\begin{equation}
\vec{v}( \vec{x} + \vec{v}( \vec{x}, t ) \epsilon + o(\epsilon) ,t + \epsilon) = \vec{v}( \vec{x},t) +  v_i( \vec{x},t ) \epsilon \frac{\partial }{\partial x_i} \vec{v}( \vec{x}, t ) + \epsilon \frac{\partial}{\partial t} \vec{v}(\vec{x},t) \epsilon. 
\end{equation}
Hence 
\begin{equation}\delta \vec{v} =  \epsilon\left( v_i   \frac{\partial }{\partial x_i} \vec{v} +  \frac{\partial}{\partial t} \vec{v}(\vec{x},t) \right) + o(\epsilon),
\label{DLVitesse} 
\end{equation}
 where all velocities in the r.h.s are now taken at point $\vec{x}$ and time $t$. The quantity we have just computed 
 is often called the {\emph{particle derivative}} of the velocity (with a differential symbol $D$ instead of $d$ or $\partial$), because it is obtained
 by following the same particle of fluid along its trajectory:
\begin{equation}
 \frac{D\vec{v}}{Dt} = \frac{\partial}{\partial t} \vec{v} + v_i   \frac{\partial }{\partial x_i} \vec{v}.
\label{particleDer}
\end{equation}
Note that it is different from the time derivative of the velocity, and that the difference is non-linear
 in the velocity field $\vec{v}$: the corresponding term is often called the convection term.
 By Cauchy's assumption and Stokes' theorem, the sum of forces acting 
 our the fluid particle is:
\begin{equation}
d\vec{F} = f^{vol}_i \vec{e_i} dV + \frac{\partial}{\partial x_j}\sigma_{ij}\vec{e_i} dV,
\end{equation}
 where $\fVol = f^{vol}_i \vec{e_i}$ represent the bulk forces applied to a unit volume (for instance
if bulk forces consist of gravity  $\fVol = - \rho g \vec{e_3}$), 
 and the equations of motion are given by:
\begin{equation}
\rho\frac{D\vec{v}}{Dt} dV=  d\vec{F},
\label{eom}
\end{equation}
 Hence, after dividing both sides by the volume element:
\begin{equation}
\boxed{
\rho\left( \frac{\partial}{\partial t} \vec{v} + \left( v_i   \frac{\partial }{\partial x_i}\right) \vec{v}\right) =  \fVol + \frac{\partial}{\partial x_j}\sigma_{ij}\vec{e_i}.}
 \label{eomGen}
 \end{equation} 
To do computations using Eq. \ref{eomGen}, we need to have a law relating the stress tensor $\sigma$ to the behaviour
of the fluid (a law that would play the role of Hooke's law in the case of fluids).\\

{\bf{Example (perfect fluids and the Euler equation).}} We will try several possibilities,
 but the simplest  law one can propose is the isotropic law corresponding to surface forces that 
 are purely normal:
\begin{equation}
\sigma_{ij}(\vec{x},t) := - P(\vec{x},t) \delta_{ij},
\label{isotropic}
\end{equation}
where $P$ is a scalar function
these are compression forces when $P$ is positive. In this case,  Eq. \ref{eomGen}
 becomes the Euler equation:
\begin{equation}
\rho\left( \frac{\partial}{\partial t} \vec{v} + \left( v_i   \frac{\partial }{\partial x_i}\right)  \vec{v}\right) = \fVol  - \frac{\partial P}{\partial x_i}\vec{e_i}.
\label{Euler}
\end{equation}
It is the equation of motions of fluids that are completely characterized by their density $\rho$ and pressure $P$. 
 These fluids are called {\emph{perfect fluids}}, or non-viscous fluids. To model the viscosity we will have to add extra terms 
 to the stress tensor and to use the more general equation of motion \ref{eomGen}.


\section{Viscous fluids}

 We want to go beyond the model of perfect fluids in order to describe the {\emph{fact}}
 that some fluids resist flow more than others.

\subsection{Material law for Newtonian fluids}

 There is no {\emph{mathematical derivation}} of a material law for fluids (there is very little theory
 when dealing with friction). 
 All we can do is to propose mathematical forms for material laws translating
 some physical assumptions, and they will have to be confronted to exepriment.
 We will limit ourselves to the simplest class of material laws, in which friction 
 is at most linear in the gradient of the velocity field (the study of fluids with more complex
 material laws is called {\emph{rheology}}).\\


 We already saw one possible material law for fluids, corresponding to an isotropic 
 stress tensor (Eq. \ref{isotropic}).
 It gives rise to Euler's equation when substituted into Eq. \ref{eomFluids},
 and it corresponds to surface  forces that are purely normal (from Cauchy's principle).
 Hence the material law \ref{isotropic} physically corresponds to neglecting 
 friction forces in the fluid (in which case the fluid is called a {\emph{perfect fluid}}).\\

 We want to keep the isotropic term in the material law, but we have 
 to include terms modeling friction.
 Newtonian fluids are fluids for which an extra term proportional to the first derivatives
 of the velocity field is included in the material law. The coefficient $\mu$ is called the 
 viscosity of the fluid, and it only depends on the nature of the fluid
  (hence perfect fluids are Newtonian fluids with zero viscosity):
\begin{equation}\boxed{
\sigma_{ij}( \vec{x}, t ) = -P(\vec{x}, t ) \delta_{ij}+ \mu\left( \frac{\partial v_i}{\partial x_j} + \frac{\partial v_j}{\partial x_i}\right).}
 \label{sigmaNewton}
\end{equation}
 In the case of thin layers of fluid, this model corresponds to friction forces that are proportional to the relative velocity 
 between thin layers of fluid. To substitute the material law (Eq. \ref{sigmaNewton}) into the equations of motion (Eq. \ref{eomFluids}), we need to compute the following derivatives for all $i$ in $\{  1,2,3\}$:
\begin{equation}
\frac{\partial\sigma_{ij}}{\partial x_j}( \vec{x}, t ) = -\frac{\partial P(\vec{x}, t )}{\partial x_j} \delta_{ij}+ \mu\left( \frac{\partial^2 v_i}{\partial x_j\partial x_j} + \frac{\partial^2 v_j}{\partial x_j\partial x_i}\right).
\end{equation}
 For incompressible fluids, the divergence of the velocity field is zero (see Eq. \ref{incomp}), so the last term in the ablove equation vanishes (after
 permuting the two derivatives), hence the 
 equations of motion become the Navier--Stokes equation:
 \begin{equation}
\boxed{
\rho\left( \frac{\partial \vec{v}}{\partial t} + ( \vec{v}.\vec{\nabla}) \vec{v}\right)= \fVol - \vec{\nabla} P + \mu \Laplace \vec{v}.
\label{NS}}
 \end{equation}

 \subsection{Boundary conditions on the velocity fields}
 The no-slipping condition is imposed: on a fixed boundary, the velocity field is zero, more 
generally the velocity of the fluid {\emph{coincides with the velocity of the boundary}} (see the 
 flow between a fixed plaque and a moving plaque for an example). This ensures that the forces 
 expressed using the stress tensor of Newtonian fluids (Eq. \ref{sigmaNewton}) do not go to infinity.








 









\end{document}




to-do: 
- biographical notices
- uniaxial

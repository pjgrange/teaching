%\documentclass[12pt]{article}
%\documentclass[a4paper, 12pt]{scrartcl}
%\voffset=-0.9in
%\hoffset=-0.6in
\documentclass[DIV=12]{article}
\setlength{\textheight}{9.1in}
\setlength{\textwidth}{7in} 
\usepackage[margin=1.1in]{geometry}
%\usepackage[justification=justified,singlelinecheck=off]{caption}
\usepackage{amsmath,esint}

\newcommand{\vol}{\mathcal{V}}
\newcommand{\surf}{{\mathcal{S}}}
\newcommand{\sExt}{{{\mathcal{S}}\rightarrow{\mathrm{ext}}}}
\newcommand{\intVol}{\iiint}
\newcommand{\intSurf}{\oiint}
\newcommand{\fVol}{\vec{f^{vol}}}
\newcommand{\eBase}{(\vec{e_1}, \vec{e_2},\vec{e_3})}
\newcommand{\eBasePrime}{\left(\vec{e'_1}, \vec{e'_2},\vec{e'_3}\right)}
\newcommand{\kg}{\mathrm{kg}}
\newcommand{\m}{\mathrm{m}}
\newcommand{\s}{\mathrm{s}}

%\cofoot{\footnotesize{\rm{Pascal Grange (August 2013)}} {\rm{\ttfamily{ pascal.grange@polytechnique.org}}}}


\let\oldbibliography\thebibliography
\renewcommand{\thebibliography}[1]{%
  \oldbibliography{#1}%
  \setlength{\itemsep}{-1pt}%
{\small
\bibliography{bibfile}}
}


\usepackage[dvips]{color}
\bibliographystyle{plain}

\usepackage{graphicx}
%\usepackage{pdfpages}
%\usepackage{multicol}
%\usepackage{pstricks,pst-grad}
%\usepackage{epsfig}
\usepackage{amsmath}

%\usepackage{subfig}
%\usepackage{tikz}
%\usepackage{verbatim}
\usepackage{amsmath}

%\usepackage{array}
%\pagestyle{scrheadings}

\begin{document}
%\maketitle
% Title
\title{
\noindent\hrulefill
\begin{flushleft}
{\Large \bf{XJTLU, MTH308 (Cartesian tensors and mathematical models of solids and viscous fluids), Semester 2, 2015\\
\vspace{8mm}
\hrule
\vspace{6mm}
 Lecture 4, 24th March, 2015: Boundary conditions, statically admissible stress tensors}}
\vspace{8mm}
\hrule
\vspace{6mm}
{\Large{Pascal Grange\\
Department of Mathematical Sciences\\
{\ttfamily{pascal.grange@xjtlu.edu.cn}}\\
}}
\noindent\hrulefill
\end{flushleft}}
\date{}
\author{}
\maketitle
%\noindent\hrulefill
\vspace{-3mm}

{\bf{Keywords}}. Interfaces, boundary conditions, hydrostatic pressure, statically admissible stress tensors.\\
\vspace{3mm}

\tableofcontents
 
\vspace{12mm}


So far we only considered continuous media that fill {\emph{the entire space}}. We studied 
 forces on {\emph{test surfaces}} inside the continuous medium. We concluded that there 
 exists a stress tensor $\sigma(\vec{x})$ at every point inside a continuous medium, and we related 
 $\sigma$ to the volume forces via a system of three PDEs.\\


We want to extend this analysis 
 to media with a boundary, and with external forces applied to this boundary (external forces). 
 This will give us boundary conditions for the PDEs.\\

 This lecture will enable us to decide whether a given stress tensor $\sigma(\vec{x})$ is 
 compatible with balance equations for a given configuration of volume and external forces.\\

 



\section{An example of stress tensor: the hydrostatic pressure}

Consider a static ocean, submitted to gravity on the surface of the 
 Earth (neglect the curvature of the Earth), assume that the density of water 
 is uniform ($\rho(\vec{x}) = 10^3  \kg.\m^{-3}$), hence the volume forces
 per unit volume in the sea are expressed as:
\begin{equation}
\fVol =-  \rho g \vec{e_3}.
\end{equation} 
where $g = 9.8 \m.\s^{-2}$, and $\eBase$ is an orthonormal base with $\vec{e_3}$ the vertical vector pointing
 upwards.\\

 The force acting on a closed test surface inside the ocean (think of a cube) is normal
 to the surface (hence the name pressure), and does not depend
 on the orientation of the surface, which we express by assuming there exists
 a positive function $P$ such that the stress tensor 
  $\sigma$  has the following form (such an isotropic diagonal tensor is sometimes called a hydrostatic
 tensor):\\
 \begin{equation}
 \sigma_{ij}(\vec{x}) n_j(\vec{x}) \vec{e_i} = -P(\vec{x}) \vec{n}, 
 \end{equation}
  where $P$ is the pressure of the water, a positive function which we have to compute.\\
 In tensor form we have $\sigma_{ij} = -P \delta_{ij}$,
 hence the equilibrium equation inside the water 
\begin{equation}
 \fVol + \frac{\partial \sigma_{ij}}{\partial x_j}\vec{e_i} = \vec{0}.
\end{equation}
 The three components of this vector equation are the following:
\[ 
 \left\{\begin{array}{l l }
0& =  - \frac{\partial P}{\partial x_1}\\
0 &= - \frac{\partial P}{\partial x_2}\\
0 & = -\rho g -\frac{\partial P}{\partial x_3}\\
\end{array}
\right.
\]
 The first two equations imply that $P$ only depends on $x_3$, and the third equation
 can be solved immediatly to show that the pressure is a linear 
 function of the depth, but we 
 need an integration constant $K$, which is the 
 value of the pressure at the surface of the water:
\begin{equation}
P(x_1,x_2,x_3 ) = -\rho g x_3 + K.
\end{equation} 
 To characterize the stress tensor completely, even in this very simple case,
 we need boundary conditions expressing the equilibrium of our continuous
 medium at the interface with other media (the atmosphere in this case),
 that can act on the medium through surface forces.
 In the next section we address this problem in full generality.


\section{Boundary conditions}

In a more general setting, let $\mathcal{V}$ be a 
 volume occupied by a contiunous medium. Its boundary 
 is denoted by $\partial \vol$. On part of the boundary, surface
 forces can be applied by an operator. Let $\vec{F}(\vec{x})$ denote the
 force by surface unit applied at a point $\vec{x}$ on the boundary $\partial\vol$.
 Let us denote by $\partial_{F}\vol$ the domain of $\partial\vol$ where 
 $\vec{F}$ is defined. The balance between volume
 forces and surface forces on $\vol$ is expressed in global form as follows:\\
 \begin{equation}
\vec{0} = \intVol_{vol} \fVol(\vec{x}) dV   + \iint_{\partial\vol -\partial_{F}\vol} \sigma_{ij}(\vec{x})n_j(\vec{x})\vec{e}_i dS+ \iint_{\partial_{F}\vol} \vec{F}(\vec{x}) dS,
 \end{equation}
 where $\vec{n}(\vec{x})$ is the unit  normal vector to the surface $\partial \vol$ at $\vec{x}$ pointing towards the exterior.
Let us add and subtract the integral of $\sigma_{ij}(\vec{x})n_j(\vec{x})\vec{e}_i $ on $\partial_F\vol$, which allows
 us to recognize an integral term on the closed surface $\partial\vol$:
\[
   \begin{array}{l l }
\vec{0}   = &   \intVol_{\vol} \fVol dV   + \iint_{\partial\vol -\partial_{F}\vol} \sigma_{ij} n_j \vec{e}_i dS +
                 \iint_{\partial_{F}\vol} \sigma_{ij} n_j(\vec{x})\vec{e}_i dS - \iint_{\partial_{F}\vol} \sigma_{ij}n_j \vec{e}_i dS  + \iint_{\partial_{F}\vol} \vec{F} dS \\
    =&  \intVol_{\vol} \fVol dV   + \intSurf_{\partial\vol\rightarrow {\mathrm{ext}}} \sigma_{ij} n_j \vec{e}_i dS - \iint_{\partial_{F}\vol} \sigma_{ij} n_j \vec{e}_i dS  + \iint_{\partial_{F}\vol} \vec{F}  dS,
   \end{array}
\]
 where for brevity the dependences of the integrands in $\vec{x}$ have not been written. 
The first two terms sum to zero, as they express the equilibrium of 
 the continuous medium in $\vol$, with an infinitely thin slice of continuous medium 
 covering it on $\partial_{F}\vol$:
\begin{equation}
\vec{0}   =   \intVol_{\vol} \fVol dV   + \intSurf_{\partial\vol\rightarrow {\mathrm{ext}} } \sigma_{ij} n_j \vec{e}_i dS,
\end{equation}
hence the equilibrium of  $\vol$ with the surface force $\vec{F}$  is written as 
\begin{equation}
\vec{0} = - \iint_{\partial_{F}\vol} \sigma_{ij} n_j \vec{e}_i dS  + \iint_{\partial_{F}\vol} \vec{F}  dS.
\end{equation}
 As this equation holds for any subset $\partial_{F}\vol$ of $\partial\vol$, it holds
 in the limit where $\partial_{F}\vol$ is an elementary surface centered on $\vec{x}$, and 
 we obtain the local boundary condition:\\
\begin{equation}
 \sigma_{ij} (\vec{x}) n_j(\vec{x}) \vec{e}_i = \vec{F}(\vec{x}),\;\;\;\forall \vec{x} \in \partial_F\vol.
 \label{boundaryCond}
\end{equation}

 {\bf{Example (hydrostatic pressure, continued).}} Let us come back to our problem of hydrostatic 
 pressure in the sea. The boundary is the entire horizontal surface  
 of equation $x_3=0$ at equilibrium with the atmosphere:
\begin{equation}
 \partial \vol = \left\{ (x_1, x_2,0), x_1 \in {\mathbf{R}}, x_2\in {\mathbf{R}}\right\}.
 \end{equation}
 Let us assume the atmospheric pressure $P_{atm}$ is uniform on the 
 surface ($P_{atm}=10^5\;{\mathrm{Pa}}$) for instance. The force on the sea per unit surface
 of the boundary is therefore:
\begin{equation}
\vec{F} = -P_{atm} \vec{e_3}.
\end{equation}
 On the boundary, the unit normal vector pointing towards the exterior of the sea is $\vec{n} = \vec{e_3}$.
 The boundary conditions \ref{boundaryCond} are therefore expressed as follows:\\
\begin{equation}
 - ( -\rho g  \times 0 + K)  \vec{e_3}  = -P_{atm} \vec{e_3}.
\end{equation}
 i.e. $K= P_{atm}$ and we find that the expression:
\begin{equation}
 P(x_1,x_2,x_3) = -\rho g x_3 + P_{atm},
\end{equation}
 and we notice that the pressure is continuous at the surface (which we 
 could have written on physical grounds without developing, this general formalism
 but we can see in this very simple case that our tensor-based 
 machinery gives the correct results).


\section{General conclusion: statically admissible stress tensors}
We can now summarize our balance laws on continuous media
  in terms of conditions on the stress tensor. If 
  a continuous medium  occupies a volume $\mathcal{V}$ and if the following 
 forces act on it:\\
-  volume forces $\fVol$ per volume unit in $\vol$,
-  specified surface forces $\vec{F}$ per surface unit on a subset $\partial_{F}\vol$ (the forces 
 may not be specified on the entire boundary $\partial_{F}\vol$),
 then the continous medium is at equilibrium if the stress tensor verified the following equations:
\[ 
 \left\{\begin{array}{l l }
\vec{0}& = \fVol(\vec{x}) + \frac{\partial \sigma_{ij}(\vec{x})}{\partial x_j} \vec{e_i},\;\;\; \forall \vec{x} \in \vol\\
\vec{F}(\vec{x}) &= \sigma_{ij} (\vec{x}) n_j(\vec{x})\vec{e_i},\;\;\;\forall \vec{x} \in \partial_{F}\vol
\end{array}
\right.
\]
where $\vec{n} = n_j( \vec{x}) \vec{e}_j$ is the normal vector to the boundary of $\vol$, pointing
 towards the exterior. \\
These equations (PDEs with boundary conditions) express the fact that the sum of forces on the 
 medium equal zero. Moreover, the tensor $\sigma$ is symmetric (which is a consequence of the 
 equilibrium of angular momenta).\\

 The set of {\emph{statically admissible stress tensors}} is defined as follows:
\begin{equation}
\boxed{\Sigma_{stat}(\vol, \fVol, \vec{F}) = \left\{   \sigma/ \sigma_{ij}(\vec{x}) =  \sigma_{ji}(\vec{x}),\;\;
\fVol(\vec{x}) + \frac{\partial \sigma_{ij}(\vec{x})}{\partial x_j} \vec{e_i} = \vec{0}\; {\mathrm{in}} \;\vol, \;{\mathrm{and}}\;
\vec{F}(\vec{x})= \sigma_{ij} n_j(\vec{x}) \vec{e_i} \;\; {\mathrm{on}}\;\; \partial_{F}\vol\right\}.}
\end{equation}
 





\end{document}
